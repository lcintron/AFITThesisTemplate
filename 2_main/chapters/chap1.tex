\section{Background and Motivation}
    Lorem ipsum dolor sit amet, consectetur adipiscing elit. Mauris nec nunc nisi. Aenean efficitur massa quam, dictum ullamcorper sapien maximus sit amet. Sed rhoncus commodo velit nec vestibulum. Integer nec nisi quis urna bibendum pharetra. Etiam posuere nunc non erat consectetur, quis aliquet leo aliquam. Aenean quis mi in enim volutpat fermentum. Vestibulum dignissim suscipit posuere. Integer lacinia non dui eu efficitur. Donec scelerisque libero condimentum imperdiet finibus. Morbi vel finibus tellus. Integer malesuada mattis dignissim. Etiam a enim a mi rhoncus eleifend quis sit amet purus.

\section{Problem Statement}
    Fusce bibendum tortor ut eros viverra imperdiet. Quisque quis arcu iaculis, volutpat metus at, semper ex. Nam ac hendrerit enim, eget fringilla urna. Curabitur quis commodo nisl, in rutrum enim. Nunc ut elit porta, volutpat mauris eget, pulvinar nisi. Proin ullamcorper accumsan erat nec dignissim. Curabitur ante lacus, sollicitudin a sapien vitae, maximus sodales nulla. Sed quis congue odio. Sed quis dui suscipit, vulputate risus id, rhoncus nisl. Pellentesque arcu mauris, egestas et urna at, rhoncus tincidunt ipsum. Sed ex magna, ornare id tellus in, scelerisque iaculis tortor.
    
\section{Examples}
\subsection{References from BibTex file}
    Use cite command and the identifier of the entry as the parameter. Bibliography is automatically generated. If doing auto-import from Mendeley, url accessed date will not be included.\\
    Single Ref:    \cite{NHTSA2010OdometerFraud} \\
    Multi Ref:     \cite{NHTSA2010OdometerFraud,NHTSA2010OdometerFraud}\\
\subsection{Acronyms}
    To use acronyms, add the acronym and definition to the acronyms.tex file under 4\_support folder.
    
    Use the following commands when using acronyms gls (regular) and glspl to show the plural form (automatically appends an `s' at the end). By using these commands, your acronyms list will be automatically generated.
    Examples:
    \begin{enumerate}
        \item  \gls{af}
        \item  \gls{af}
        \item  \glspl{af}
    \end{enumerate}
    
    
\subsection{Tables}
    Tables can be defined in the tables.tex folder and referenced here by command name or within the chapter file as follows:
    \begin{table}[!h]
	\centering
	\small
	\rowcolors{2}{gray!25}{white}
	\caption{DLT Alternatives.}\label{table:DLTComparison}
	\begin{tabular}{>{\centering\arraybackslash}p{1.9cm}>{\centering\arraybackslash}p{1.9cm}>{\centering\arraybackslash}p{1.9cm}>{\centering\arraybackslash}p{2.2cm}>{\centering\arraybackslash}p{2.2cm}>{\centering\arraybackslash}p{2cm}}
		\toprule
		\textbf{Name}&\textbf{Executable}&\textbf{API/SDK Ready}&\textbf{Permissioned}&\textbf{Smart Contracts}&\textbf{\$0 License Cost} \\
		\midrule
		Tendermint & \checkmark  & \checkmark  & &  & \checkmark \\
		MultiChain & \checkmark & \checkmark& \checkmark &\checkmark  &  \\
		Hyperledger Fabric & \checkmark & \checkmark& \checkmark &  \checkmark & \checkmark \\
		\bottomrule
	\end{tabular}
    \end{table}
    
\subsection{Figures}
    Figures can be defined in the figures.tex file and referenced here by command name or within the chapter file as follows:
    \figtitlePage
    
    \begin{figure}[tbp]
        \vspace{20pt}
        \begin{center}
            \includegraphics[width=2in]{5_figures/afitlogo}
            \caption{Enter student data in titlePage.tex to customize the document's first pages.}
         \label{fig:titlePageTwo}
     \end{center}
    \end{figure}

\subsection{Algorithms}
    Example of algorithmic package for algorithms.
    
    \begin{algorithm}[!h]
        	\caption{Road Event Witnessed Data Transaction}
        	\begin{algorithmic}[1]
        		\State //tx $\leftarrow$ \{sourceId, eventId, WitnessedData\}
        		\If{tx.WitnessedData.length$>0$ \& SensorExists(tx.sourceId) \& IsValid(tx)} %RoadEventExists(tx.eventId)
        		\State $wd \leftarrow WitnessedData(tx)$
        		\State $WitnessedDataRegistry.add(wd)$
        		\State emit(WitnessedDataSubmitted)
        		\Else
        		\State emit(InvalidWitnessedDataTx)
        		\EndIf
        	\end{algorithmic}
        	\label{algo:submitWitness}
        \end{algorithm}

\subsection{Appendices}
    Defined in 2\_main/appendix and imported to main.tex file. Reference Appendix~\ref{appendix:configuration}.

