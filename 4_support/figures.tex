%% myFigures.tex
% A common file to store all figure definitions
%
% In preparing your thesis, one of the first things you should do is
% organize your figures.  Then, one of the last things you'll do is
% reorder your figures so they display where you want them to in the
% text.  Organizing figure definitions in a common files helps:
%
%   1. Write new figures using earlier examples.
%
%   2.  Isolate code and minimize the risk of introducing bugs in the
%   final editing process.  Trust me, moving around just one line of
%   code is easier.
%
%   3.  Reuse figures in other papers.  <=== the best reason!
%
% Note command names can not include numbers and special characters.
%
% To make the file more searchable, use naming conventions that map
% the graphics filename labSetup.jpg to the command name \figlabSetup to the
% figure label fig:labSetup.
% 

%\graphicspath{/5_figures/}
%%%%%%%%%%%%%%%%%%%%%%%%%%%%%%%%%%%%%%%%%%%
\newcommand{\figTransporationConsortium}{\begin{figure}[h]
    \vspace{20pt}
    \begin{center}
    	\includegraphics[width=250pt]{5_figures/ConsortiumFigure.png}
    	\caption{Transportation Infrastructure Consortium Members} \label{figTransporationConsortium}
    \end{center}
\end{figure}
}


\newcommand{\figtitlePage}{\begin{figure}[tbp]
    \vspace{20pt}
    \begin{center}
        \includegraphics[width=2in]{5_figures/afitlogo}
        \caption{Enter student data in titlePage.tex to customize the document's first pages.}
     \label{fig:titlePage}
 \end{center}
\end{figure}
}
